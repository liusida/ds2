\newcommand\chapternumber{2}
\documentclass[12pt,a4paper]{article}
\usepackage{fullpage}
\usepackage[top=2cm, bottom=4.5cm, left=2.5cm, right=2.5cm]{geometry}
\usepackage{amsmath,amsthm,amsfonts,amssymb,amscd}
\usepackage{lastpage}
\usepackage{enumitem}
\usepackage{fancyhdr}
\usepackage{mathrsfs}
\usepackage{xcolor}
\usepackage{graphicx}
\usepackage{listings}
\usepackage{hyperref}
\usepackage{tikz}
\usetikzlibrary{shapes,backgrounds}
\usepackage[utf8]{inputenc}
\usepackage[ruled, vlined]{algorithm2e}

% Edit these as appropriate
\newcommand\course{Data Science II}
\newcommand\NetID{sliu1@uvm.edu}
\newcommand\Author{Sida Liu}
\pagestyle{fancyplain}
\headheight 35pt
\lhead{\NetID\\\Author}
\chead{\textbf{\Large Assignment \chapternumber }}
\rhead{\course \\ \today}
\lfoot{}
\cfoot{}
\rfoot{\small\thepage}
\headsep 1.5em

\setlength{\parskip}{\baselineskip}%
\setlength{\parindent}{0pt}%

\newenvironment{list_abc}
{ \begin{enumerate}[label=(\alph*)] }
{ \end{enumerate} }

\newenvironment{list_iv}
{ \begin{enumerate}[label=\roman*.] }
{ \end{enumerate} }

\hypersetup{
    colorlinks=true,
    linkcolor=blue,
    filecolor=magenta,      
    urlcolor=cyan,
}
\usepackage{enumerate}
\usepackage{float}

\begin{document}

\section{Find constants / partition functions.}
\subsection{}
\begin{align*}
    f(x|a,b) = 1 \text{ on } x \in [a,b]
\end{align*}

First, let's take integral of $f(x)$:
\begin{align*}
    \int f(x) \ dx
    &= \int_a^b 1 \ dx\\
    &= b-a
\end{align*}

So, if we want $\int p(x) \ dx = 1$, and $p(x) = \frac{1}{\mathcal{Z}} f(x)$, we have:
\begin{align}
    \int \frac{1}{\mathcal{Z}} f(x) \ dx&= 1 \nonumber \\
    \frac{1}{\mathcal{Z}} \int  f(x) \ dx&= 1 \nonumber \\
    \int f(x) \ dx&= \mathcal{Z}  \\
    b-a &= \mathcal{Z} \nonumber 
\end{align}

So, $\mathcal{Z} = b-a$. 

Equation (1) can also apply to other functions in question 1, and it indicates that the normalizing constant is just the integral of the function.

\subsection{}
\begin{align*}
    f(x|\beta) = \exp(-\frac{x}{\beta}) \text{ on } x \in (0,\infty) \text{ and } \beta \in (0, \infty)
\end{align*}

First, let's take integral of $f(x)$:
\begin{align*}
    \int f(x) \ dx
    &= \int_0^{\infty} \exp(-\frac{x}{\beta})\ dx \\
    &= (-\beta \cdot \exp(-\frac{x}{\beta})) \biggr\rvert _0^{\infty} \\
    &= -\beta \cdot (\exp(-\infty)-\exp(0)) \\
    &= \beta 
\end{align*}

So, substitude to equation (1), we have $\mathcal{Z} = \beta$.

\subsection{}
\begin{align*}
    f(x|\mu,\sigma)=\exp[-\frac{(x-\mu)^2}{2\sigma^2}]
\end{align*}
where $x \in (-\infty, \infty)$, $\mu \in (-\infty, \infty)$, $\sigma \in (0, \infty)$.

To find the integral of $f$, we need to square it:
\begin{align*}
    \int f(x|\mu, \sigma) \ dx &= \sqrt{\int_{-\infty}^{\infty} f(x|\mu, \sigma) \ dx \cdot \int_{-\infty}^{\infty} f(x|\mu, \sigma) \ dx} \\
    &= \sqrt{\int_{-\infty}^{\infty} f(x|\mu, \sigma) \ dx \cdot \int_{-\infty}^{\infty} f(y|\mu, \sigma) \ dy} \\
    &= \sqrt{\int_{-\infty}^{\infty} \int_{-\infty}^{\infty} f(x|\mu, \sigma) \cdot f(y|\mu, \sigma) \ dx\ dy} \\
    &= \sqrt{\int_{-\infty}^{\infty} \int_{-\infty}^{\infty} \exp[-\frac{(x-\mu)^2}{2\sigma^2}] \cdot \exp[-\frac{(y-\mu)^2}{2\sigma^2}] \ dx\ dy} \\
    &= \sqrt{\int_{-\infty}^{\infty} \int_{-\infty}^{\infty} \exp[-\frac{(x-\mu)^2}{2\sigma^2}-\frac{(y-\mu)^2}{2\sigma^2}] \ dx\ dy} \\
    &= \sqrt{\int_{-\infty}^{\infty} \int_{-\infty}^{\infty} \exp[-\frac{(x-\mu)^2 + (y-\mu)^2}{2\sigma^2}] \ dx\ dy} \\
\end{align*}

Let's set the new origin at $(\mu, \mu)$, and turn the equation into polar coordinate:
\begin{align*}
    \int f(x|\mu, \sigma) \ dx
    &= \sqrt{\int_{-0}^{2 \pi} \int_{0}^{\infty} \exp[-\frac{r^2}{2\sigma^2}] \cdot r \ dx\ dy} \\
    &= \sqrt{\int_{-0}^{2 \pi} \sigma^2 \ dy} \\
    &= \sqrt{2 \sigma^2 \pi } \\
\end{align*}

We know $\sigma > 0$, so $\mathcal{Z} = \sigma \sqrt{2 \pi}$.
% First, we start from the PDF of the Normal distribution: 
% \begin{align*}
%     p(x) &= \frac{1}{\sigma \sqrt{2 \pi}} \cdot \exp[-\frac{(x-\mu)^2}{2\sigma^2}] \\
%     &= \frac{1}{\sigma \sqrt{2 \pi}} \cdot f(x|\mu, \sigma)
% \end{align*}

% So,
% \begin{align*}
%     f(x|\mu,\sigma) = \sigma \sqrt{2 \pi} \cdot p(x)
% \end{align*}

% We know that Normal distribution is a probability distribution, so $\int p(x) \ dx=1$. 
% Thus $\int f(x|\mu, \sigma) \ dx  = \sigma \sqrt{2 \pi} \cdot 1$, and $\mathcal{Z} = \sigma \sqrt{2 \pi}$.
\subsection{}
\begin{align*}
    f(x|a,\gamma)=\frac{1}{x^\gamma}
\end{align*}
where $x \in (a, \infty)$, $a \in (0, \infty)$, $\gamma \in (0, \infty)$.

There are three cases: (1) $\gamma>1$, (2) $\gamma=1$, (3) $0<\gamma<1$.

In case (1) $\gamma>1$:
\begin{align*}
    \int_a^{\infty} \frac{1}{x^{\gamma}} \ dx
    &= \frac{x^{1-\gamma}}{1-\gamma} \biggr\rvert _a^{\infty} \\
    &= 0 - \frac{a^{1-\gamma}}{1-\gamma} \\
    &= \frac{1}{a^{\gamma-1} (\gamma-1)}
\end{align*}

So, in case (1), $\mathcal{Z} = \frac{1}{a^{\gamma-1} (\gamma-1)}$.

In case (2) $\gamma=1$:
\begin{align*}
    \int_a^{\infty} \frac{1}{x} \ dx
    &= \log(x) \biggr\rvert _a^{\infty} \\
    &= \infty
\end{align*}

So, in case (2), $\mathcal{Z}$ doesn't exist.

In case (3) $0<\gamma<1$:
\begin{align*}
    \int_a^{\infty} \frac{1}{x^{\gamma}} \ dx
    &= \frac{x^{1-\gamma}}{1-\gamma} \biggr\rvert _a^{\infty} \\
    &= \infty    
\end{align*}

So, in case (3), $\mathcal{Z}$ doesn't exist.

Thus, in summary, if $f(x|a,\gamma)$ is defined on $\gamma \in (0, \infty)$, $\mathcal{Z}$ doesn't exist.

\section{1st and 2nd moment}
In the previous question, only 1.1 and 1.2 can be properly-normalized.

\subsection{}
For 1.1, $\mathcal{Z}=b-a$, and $p(x)=\frac{1}{b-a}$.
\begin{align*}
    m_1 &= E_{x \sim p(x)}[x] \\
    &= \int_a^b p(x) x \ dx\\
    &= \int_a^b \frac{x}{b-a} \ dx\\
    &= \frac{a+b}{2}
\end{align*}

\begin{align*}
    m_2 &= E_{x \sim p(x)}[(x-m_1)^2] \\
    &= E_{x \sim p(x)}[(x-\frac{a+b}{2})^2] \\
    &= \int_a^b p(x) (x-\frac{a+b}{2})^2 \ dx\\
    &= \int_a^b \frac{(x-\frac{a+b}{2})^2}{b-a}  \ dx\\
    &= \frac{1}{b-a} \int_a^b (x^2 - (a+b)x + \frac{1}{4}(a+b)^2) )  \ dx\\
    &= \frac{1}{b-a} (\frac{1}{3}(b^3-a^3) - \frac{1}{2}(a+b)(b^2-a^2) + \frac{1}{4}(b-a)(a+b)^2) \\
    &= \frac{1}{12} (b-a)^2
\end{align*}

\subsection{}
For 1.2, $\mathcal{Z}=\beta$, and $p(x)=\frac{1}{\beta} \exp(-\frac{x}{\beta})$.
\begin{align*}
    m_1 &= E_{x \sim p(x)}[x] \\
    &= \int_0^{\infty} p(x) x \ dx\\
    &= \int_0^{\infty} \frac{1}{\beta} \exp(-\frac{x}{\beta}) x \ dx\\
    &= \frac{1}{\beta} \int_0^{\infty} \exp(-\frac{x}{\beta}) x \ dx\\
    &= \frac{1}{\beta} [\beta (-\exp(-\frac{x}{\beta})(\beta+x))] \biggr\rvert _0^{\infty} \\
\end{align*}
Here, we have $\beta>0$, so, $\exp(-\frac{x}{\beta}) x \rightarrow 0$ when $x \rightarrow 0$.
\begin{align*}
    m_1 &= \frac{1}{\beta} [0 + \beta (\beta+0))] \\
    &= \beta
\end{align*}

\begin{align*}
    m_2 &= E_{x \sim p(x)}[(x-m_1)^2] \\
    &= E_{x \sim p(x)}[(x-\beta)^2] \\
    &= \int_0^{\infty} p(x) (x-\beta)^2 \ dx \\
    &= \int_0^{\infty} \frac{1}{\beta} \exp(-\frac{x}{\beta}) (x-\beta)^2 \ dx \\
    &= \frac{1}{\beta} \int_0^{\infty} \exp(-\frac{x}{\beta}) (x^2-2\beta x + \beta^2) \ dx \\
    &= \frac{1}{\beta} [ \int_0^{\infty} \exp(-\frac{x}{\beta}) x^2 \ dx -2 \int_0^{\infty} \exp(-\frac{x}{\beta}) \beta x \ dx + \int_0^{\infty} \exp(-\frac{x}{\beta}) \beta^2 \ dx ] \\
    &= \frac{1}{\beta} [ (\beta(-\exp(-\frac{x}{\beta})(2\beta^2+x^2+2\beta x)))|_0^{\infty} -2 (\beta^2(-\exp(-\frac{x}{\beta})(\beta+x)))|_0^{\infty} \\
    &\  + (\beta^3(-\exp(-\frac{x}{\beta})))|_0^{\infty} ]
\end{align*}
For the same argument in case of $\beta>0$, $\exp(-\frac{x}{\beta}) x \rightarrow 0$ and $\exp(-\frac{x}{\beta}) x^2 \rightarrow 0$ when $x \rightarrow 0$:
\begin{align*}
    m_2 &= \frac{1}{\beta} [ 2 \beta^3 -2 \beta^3 + \beta^3 ] \\
    &= \beta^2
\end{align*}

\subsection{}
For 1.3, $\mathcal{Z} = \sigma \sqrt{2 \pi}$, and $p(x)=\frac{1}{\sigma \sqrt{2 \pi}} \exp(-\frac{(x-\mu)^2}{2 \sigma^2})$.
\begin{align*}
    m_1 &= E_{x \sim p(x)}[x] \\
    &= \int_{-\infty}^{\infty} p(x) x \ dx \\
    &= \int_{-\infty}^{\infty} \frac{1}{\sigma \sqrt{2 \pi}} \exp(-\frac{(x-\mu)^2}{2 \sigma^2}) x \ dx \\
    &= \int_{-\infty}^{\infty} \frac{1}{\sigma \sqrt{2 \pi}} \exp(-\frac{(x-\mu)^2}{2 \sigma^2}) (x-\mu) + \mu \cdot \frac{1}{\sigma \sqrt{2 \pi}} \exp(-\frac{(x-\mu)^2}{2 \sigma^2}) \ dx \\
\end{align*}

Let $y = x - \mu$, we can see the first part is 0 (from question 1.3).
And we write the second part back into the form of $p(x)$, which is the normalized probability density function.
\begin{align*}
    m_1
    &= \int_{-\infty}^{\infty} \frac{1}{\sigma \sqrt{2 \pi}} \exp(-\frac{y^2}{2 \sigma^2}) y + \mu \cdot p(x) \ dx \\
    &= \int_{-\infty}^{\infty} \mu \cdot p(x) \ dx \\
    &= \mu
\end{align*}

\begin{align*}
    m_2 &= E_{x \sim p(x)}[(x-m_1)^2] \\
    &= E_{x \sim p(x)}[(x-\mu)^2] \\
    &= \int_{-\infty}^{\infty} p(x) (x-\mu)^2 \ dx \\
    &= \int_{-\infty}^{\infty} \frac{1}{\sigma \sqrt{2 \pi}} \exp[-\frac{(x-\mu)^2}{2 \sigma^2}] (x-\mu)^2 \ dx \\
\end{align*}

Let $y=\frac{x-\mu}{\sqrt{2}\sigma}$, so, $dx = \sqrt{2} \sigma\ dy$, we have:
\begin{align*}
    m_2 &= 
    \int_{-\infty}^{\infty} \frac{1}{\sigma \sqrt{2 \pi}} \exp[-y^2] 2\sigma^2 y^2  \sqrt{2} \sigma \ dy \\
    &= \sigma^2 \int_{-\infty}^{\infty} \frac{2}{\sqrt{\pi}} \exp[-y^2]  y^2  \ dy \\
\end{align*}

We know from question 1, $\int_{-\infty}^{\infty} \exp[-\frac{(x-\mu)^2}{2\sigma^2}] \ dx = \sigma \sqrt{2 \pi}$.
If we differentiate both sides w.r.t. $y$, we have :
\begin{align*}
    \int_{-\infty}^{\infty} 2 y^2 \exp[-y^2] \sigma \sqrt{2} \ dy &= \sigma \sqrt{2 \pi} \\
    \int_{-\infty}^{\infty} \frac{2}{\sqrt{\pi}} \exp[-y^2]  y^2  \ dy &= 1
\end{align*}

Substitude back in, we have:
\begin{align*}
    m_2 &= \sigma^2
\end{align*}

\subsection{}
For 1.4, first, we have:
\begin{align*}
    m_1 &= E_{x \sim p(x)}[x] \\
    &= \int_{a}^{\infty} p(x) x \ dx \\
    &= \frac{1}{\mathcal{Z}} \int_{a}^{\infty} \frac{1}{x^\gamma} x \ dx \\
    &= \frac{1}{\mathcal{Z}} \int_{a}^{\infty} x^{1-\gamma} \ dx \\
    &= \frac{1}{\mathcal{Z}} \frac{1}{2-\gamma} [x^{2-\gamma}] \biggr\rvert_a^{\infty} \\
\end{align*}

We can see, (1) in the case of $\gamma<=2$, $m_1 \rightarrow \infty$; 

(2) in the case of $\gamma>2$, $x^{2-\gamma} |_a^{\infty}=-a^{2-\gamma}$, so:
\begin{align*}
    m_1 
    &= a^{\gamma-1} (\gamma-1) \frac{1}{2-\gamma} (-a^{2-\gamma}) \\
    &= \frac{\gamma-1}{\gamma-2} \cdot a \\
\end{align*}

Now in order to calculate $m_2$, we need $m_1$ to be finite, (and for case $\gamma<=2$ we don't have finite $m_2$ ):
\begin{align*}
    m_2 &= E_{x \sim p(x)}[(x-m_1)^2] \\
    &= \frac{1}{\mathcal{Z}} \int_{a}^{\infty} x^{-\gamma} (x-m_1)^2 \ dx \\
    &= \frac{1}{\mathcal{Z}} \int_{a}^{\infty} x^{-\gamma} (x^2 - 2 m_1 x + m_1 ^2) \ dx \\
    &= \frac{1}{\mathcal{Z}} \int_{a}^{\infty} (x^{2-\gamma} - 2 m_1 x^{1-\gamma} + m_1 ^2 x^{-\gamma}) \ dx \\
\end{align*}

For the same argument when calculating $m_1$, we can see, (1) in the case of $\gamma<=3$, $m_2 \rightarrow \infty$;

(2) in the case of $\gamma>3$, we have:
\begin{align*}
    m_2 &= \frac{1}{\mathcal{Z}} \int_{a}^{\infty} (x^{2-\gamma} - 2 m_1 x^{1-\gamma} + m_1 ^2 x^{-\gamma}) \ dx \\
    &= a^{\gamma-1} (\gamma-1) \biggr[ \int_{a}^{\infty} (x^{2-\gamma} - 2 m_1 x^{1-\gamma} + m_1 ^2 x^{-\gamma}) \ dx \biggr] \\
    &= a^{\gamma-1} (\gamma-1) \biggr[ \int_{a}^{\infty} x^{2-\gamma} \ dx - 2 m_1 \int_{a}^{\infty} x^{1-\gamma} \ dx + m_1 ^2 \int_{a}^{\infty} x^{-\gamma} \ dx \biggr] \\
    &= a^{\gamma-1} (\gamma-1) \biggr[ \frac{a^{3-\gamma}}{\gamma-3} - 2 m_1 \frac{a^{2-\gamma}}{\gamma-2} + m_1 ^2 \frac{a^{1-\gamma}}{\gamma-1} \biggr] \\
    &= a^{\gamma-1} (\gamma-1) \biggr[ \frac{a^{3-\gamma}}{\gamma-3} - 2 \frac{\gamma-1}{\gamma-2} \cdot a \frac{a^{2-\gamma}}{\gamma-2} + (\frac{\gamma-1}{\gamma-2} \cdot a)^2 \frac{a^{1-\gamma}}{\gamma-1} \biggr] \\
    &= a^2 (\gamma-1) \biggr[ \frac{1}{\gamma-3} - 2 \frac{\gamma-1}{\gamma-2} \frac{1}{\gamma-2} + (\frac{\gamma-1}{\gamma-2})^2 \frac{1}{\gamma-1} \biggr] \\
    &= a^2 (\gamma-1) \biggr[ \frac{1}{\gamma-3} - 2 \frac{\gamma-1}{(\gamma-2)^2} + \frac{\gamma-1}{(\gamma-2)^2} \biggr] \\
    &= a^2 (\gamma-1) \biggr[ \frac{1}{\gamma-3} - \frac{\gamma-1}{(\gamma-2)^2} \biggr] \\
    &= a^2 \biggr[ \frac{\gamma-1}{\gamma-3} - \frac{(\gamma-1)^2}{(\gamma-2)^2} \biggr] \\
\end{align*}

\subsection*{Comments}
When analyzing data, we can compute the 1st and 2nd moment of the samples, and guess the distribution according to them.

For example, if we found $m_2 = m_1^2$, it might suggest that the data is from an exponential distribution.

\section{Simulation}


    \newpage
    \begin{table}[h!]
        \begin{center} {\footnotesize
        \begin{tabular}{ccccc}
        \hline
        & \multicolumn{1}{c}{ Uniform }  \\
        \multicolumn{1}{c}{params} & \multicolumn{1}{c}{N} & \multicolumn{1}{c}{m1} & \multicolumn{1}{c}{m1}\\
        \hline
        
(0, 1) & N=1 & 0.581 & 0.000 \\
(0, 1) & N=10 & 0.485 & 0.072 \\
(0, 1) & N=100 & 0.505 & 0.085 \\
(0, 1) & N=1000 & 0.497 & 0.084 \\
(0, 1) & N=10000 & 0.500 & 0.083 \\
(0, 1) & truth & 0.500 & 0.083 \\
 & Compare to truth & 1.000 & 1.000 \\
\\
(3, 7) & N=1 & 5.080 & 0.000 \\
(3, 7) & N=10 & 5.032 & 1.259 \\
(3, 7) & N=100 & 5.003 & 1.339 \\
(3, 7) & N=1000 & 4.997 & 1.337 \\
(3, 7) & N=10000 & 5.002 & 1.332 \\
(3, 7) & truth & 5.000 & 1.333 \\
 & Compare to truth & 1.000 & 1.001 \\
\\
(11, 39) & N=1 & 24.066 & 0.000 \\
(11, 39) & N=10 & 26.351 & 53.341 \\
(11, 39) & N=100 & 25.131 & 64.283 \\
(11, 39) & N=1000 & 25.054 & 65.170 \\
(11, 39) & N=10000 & 25.027 & 65.336 \\
(11, 39) & truth & 25.000 & 65.333 \\
 & Compare to truth & 0.999 & 1.000 \\
\\
(47, 133) & N=1 & 88.355 & 0.000 \\
(47, 133) & N=10 & 89.945 & 513.608 \\
(47, 133) & N=100 & 90.216 & 598.534 \\
(47, 133) & N=1000 & 90.136 & 618.233 \\
(47, 133) & N=10000 & 89.957 & 616.293 \\
(47, 133) & truth & 90.000 & 616.333 \\
 & Compare to truth & 1.000 & 1.000 \\
\\
\hline
        \end{tabular} }
        \end{center}
        \label{turns}
    \end{table}
    

    \newpage
    \begin{table}[h!]
        \begin{center} {\footnotesize
        \begin{tabular}{ccccc}
        \hline
        & \multicolumn{1}{c}{ Exponential }  \\
        \multicolumn{1}{c}{params} & \multicolumn{1}{c}{N} & \multicolumn{1}{c}{m1} & \multicolumn{1}{c}{m1}\\
        \hline
        
1 & N=1 & 1.337 & 0.000 \\
1 & N=10 & 1.044 & 1.199 \\
1 & N=100 & 1.011 & 1.027 \\
1 & N=1000 & 1.008 & 1.046 \\
1 & N=10000 & 1.000 & 1.003 \\
1 & truth & 1.000 & 1.000 \\
 & Compare to truth & 1.000 & 0.997 \\
\\
3.7 & N=1 & 3.541 & 0.000 \\
3.7 & N=10 & 3.416 & 11.375 \\
3.7 & N=100 & 3.814 & 13.879 \\
3.7 & N=1000 & 3.716 & 14.106 \\
3.7 & N=10000 & 3.702 & 13.708 \\
3.7 & truth & 3.700 & 13.690 \\
 & Compare to truth & 0.999 & 0.999 \\
\\
11.39 & N=1 & 12.311 & 0.000 \\
11.39 & N=10 & 11.058 & 106.050 \\
11.39 & N=100 & 11.179 & 121.437 \\
11.39 & N=1000 & 11.342 & 129.147 \\
11.39 & N=10000 & 11.359 & 129.765 \\
11.39 & truth & 11.390 & 129.732 \\
 & Compare to truth & 1.003 & 1.000 \\
\\
47.133 & N=1 & 43.559 & 0.000 \\
47.133 & N=10 & 49.199 & 2231.948 \\
47.133 & N=100 & 47.318 & 2095.584 \\
47.133 & N=1000 & 47.266 & 2197.766 \\
47.133 & N=10000 & 47.137 & 2212.673 \\
47.133 & truth & 47.133 & 2221.520 \\
 & Compare to truth & 1.000 & 1.004 \\
\\
\hline
        \end{tabular} }
        \end{center}
        \label{turns}
    \end{table}
    

    \newpage
    \begin{table}[h!]
        \begin{center} {\footnotesize
        \begin{tabular}{ccccc}
        \hline
        & \multicolumn{1}{c}{ Normal }  \\
        \multicolumn{1}{c}{params} & \multicolumn{1}{c}{N} & \multicolumn{1}{c}{m1} & \multicolumn{1}{c}{m1}\\
        \hline
        
(0, 1) & N=1 & 0.143 & 0.000 \\
(0, 1) & N=10 & 0.069 & 0.904 \\
(0, 1) & N=100 & -0.001 & 1.010 \\
(0, 1) & N=1000 & -0.004 & 0.999 \\
(0, 1) & N=10000 & -0.000 & 1.002 \\
(0, 1) & truth & 0.000 & 1.000 \\
 & Compare to truth & -0.000 & 0.998 \\
\\
(3, 7) & N=1 & 2.112 & 0.000 \\
(3, 7) & N=10 & 3.343 & 41.159 \\
(3, 7) & N=100 & 3.043 & 45.031 \\
(3, 7) & N=1000 & 2.988 & 48.990 \\
(3, 7) & N=10000 & 2.988 & 48.931 \\
(3, 7) & truth & 3.000 & 49.000 \\
 & Compare to truth & 1.004 & 1.001 \\
\\
(11, 39) & N=1 & 23.553 & 0.000 \\
(11, 39) & N=10 & 9.021 & 1111.904 \\
(11, 39) & N=100 & 11.533 & 1478.462 \\
(11, 39) & N=1000 & 11.216 & 1529.614 \\
(11, 39) & N=10000 & 10.950 & 1513.590 \\
(11, 39) & truth & 11.000 & 1521.000 \\
 & Compare to truth & 1.005 & 1.005 \\
\\
(47, 133) & N=1 & 18.069 & 0.000 \\
(47, 133) & N=10 & 51.814 & 16865.779 \\
(47, 133) & N=100 & 45.570 & 16630.440 \\
(47, 133) & N=1000 & 47.869 & 17548.646 \\
(47, 133) & N=10000 & 46.787 & 17644.365 \\
(47, 133) & truth & 47.000 & 17689.000 \\
 & Compare to truth & 1.005 & 1.003 \\
\\
\hline
        \end{tabular} }
        \end{center}
        \label{turns}
    \end{table}
    

    \newpage
    \begin{table}[h!]
        \begin{center} {\footnotesize
        \begin{tabular}{ccccc}
        \hline
        & \multicolumn{1}{c}{ Power Law }  \\
        \multicolumn{1}{c}{params} & \multicolumn{1}{c}{N} & \multicolumn{1}{c}{m1} & \multicolumn{1}{c}{m1}\\
        \hline
        
(1.5, 1.5) & N=1 & 726.346 & 0.000 \\
(1.5, 1.5) & N=10 & 197.476 & 756536.277 \\
(1.5, 1.5) & N=100 & 11829.485 & 143833701075.251 \\
(1.5, 1.5) & N=1000 & 170063.068 & 599332955437829.750 \\
(1.5, 1.5) & N=10000 & 15768315.467 & 34687120909817544704.000 \\
(1.5, 1.5) & truth & inf & inf \\
 & Compare to truth & inf & inf \\
\\
(2.2, 2.2) & N=1 & 5.564 & 0.000 \\
(2.2, 2.2) & N=10 & 8.538 & 126.861 \\
(2.2, 2.2) & N=100 & 10.133 & 3505.338 \\
(2.2, 2.2) & N=1000 & 11.613 & 15444.646 \\
(2.2, 2.2) & N=10000 & 14.812 & 3348093.832 \\
(2.2, 2.2) & truth & 13.200 & inf \\
 & Compare to truth & 0.891 & inf \\
\\
(3.3, 3.3) & N=1 & 6.814 & 0.000 \\
(3.3, 3.3) & N=10 & 5.947 & 22.821 \\
(3.3, 3.3) & N=100 & 5.950 & 32.098 \\
(3.3, 3.3) & N=1000 & 5.860 & 42.133 \\
(3.3, 3.3) & N=10000 & 5.824 & 41.184 \\
(3.3, 3.3) & truth & 5.838 & 49.402 \\
 & Compare to truth & 1.003 & 1.200 \\
\\
(4.4, 4.4) & N=1 & 6.013 & 0.000 \\
(4.4, 4.4) & N=10 & 6.338 & 10.597 \\
(4.4, 4.4) & N=100 & 6.147 & 6.562 \\
(4.4, 4.4) & N=1000 & 6.215 & 7.091 \\
(4.4, 4.4) & N=10000 & 6.233 & 8.307 \\
(4.4, 4.4) & truth & 6.233 & 8.163 \\
 & Compare to truth & 1.000 & 0.983 \\
\\
\hline
        \end{tabular} }
        \end{center}
        \label{turns}
    \end{table}
    



\subsection*{Convergence}
The estimated $m_1$ and $m_2$ converge to analytical solutions of $m_1$ and $m_2$.
The variance of $m_1$ and $m_2$ decrease with $K$.
Yes, the simulation confirms the analytical solutions are good.


\section{Derivation from (11) to (12)}
Because the constraints are:
\begin{align*}
    E_{x \sim p(x)}[f_i(x)] = F_i
\end{align*}
for $1 \leq i \leq I$, let's write the function $J$ in this way:
\begin{align*}
    J = \sum_x p(x) \log \frac{1}{p(x)} - \sum_{1 \leq i \leq I} \lambda_i (E_{x \sim p(x)}[f_i(x)]-F_i)
\end{align*}
And we start to solve for $\partial_{p(x)} J = 0$, we have:
\begin{align*}
    \partial_{p(x)} \biggr[ \sum_x p(x) \log \frac{1}{p(x)} - \sum_{1 \leq i \leq I} \lambda_i \biggr(E_{x \sim p(x)}[f_i(x)]-F_i\biggr) \biggr] &= 0 \\
    \partial_{p(x)} \biggr[ \sum_x p(x) \log \frac{1}{p(x)} - \sum_{1 \leq i \leq I} \lambda_i \biggr(\sum_{x} p(x) f_i(x)-F_i\biggr) \biggr] &= 0 \\
    \sum_x \partial_{p(x)} p(x) \log \frac{1}{p(x)} - \sum_{1 \leq i \leq I} \partial_{p(x)} \lambda_i \biggr(\sum_{x} p(x) f_i(x)-F_i\biggr) &= 0 \\
    \sum_x \partial_{p(x)} p(x) \log \frac{1}{p(x)} - \sum_{1 \leq i \leq I} \partial_{p(x)} \lambda_i \sum_{x} p(x) f_i(x) - \sum_{1 \leq i \leq I} \partial_{p(x)} \lambda_i F_i &= 0 \\
    \sum_x \partial_{p(x)} p(x) \log \frac{1}{p(x)} - \sum_x \partial_{p(x)} p(x) \sum_{1 \leq i \leq I} \lambda_i f_i(x) - 0 &= 0 \\
    \sum_x \partial_{p(x)} p(x) \biggr[ \log \frac{1}{p(x)} - \sum_{1 \leq i \leq I} \lambda_i f_i(x) \biggr] &= 0 \\
\end{align*}
Because in general $\partial_{p(x)} p(x) \neq 0$, so:
\begin{align*}
    \log \frac{1}{p(x)} - \sum_{1 \leq i \leq I} \lambda_i f_i(x) &= 0 \\
    \log p(x) &= - \sum_{1 \leq i \leq I} \lambda_i f_i(x)  \\
    p(x) &= \exp \biggr( - \sum_{1 \leq i \leq I} \lambda_i f_i(x) \biggr) \\
\end{align*}
Though ``$\propto$'' is used in the lecture note, but since $\lambda_i$ can be any constants, I think it is OK to use ``$=$''.

\section{Prove there is a maximum entropy distribution on R}
Proof:

The Normal distribution is a maximum entropy distribution, and the Normal distribution has a support of $R$,
thus there is a maximum entropy distribution on $R$.

\end{document}