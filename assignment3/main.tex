\newcommand\chapternumber{3}
\documentclass[12pt,a4paper]{article}
\usepackage{fullpage}
\usepackage[top=2cm, bottom=4.5cm, left=2.5cm, right=2.5cm]{geometry}
\usepackage{amsmath,amsthm,amsfonts,amssymb,amscd}
\usepackage{lastpage}
\usepackage{enumitem}
\usepackage{fancyhdr}
\usepackage{mathrsfs}
\usepackage{xcolor}
\usepackage{graphicx}
\usepackage{listings}
\usepackage{hyperref}
\usepackage{tikz}
\usetikzlibrary{shapes,backgrounds}
\usepackage[utf8]{inputenc}
\usepackage[ruled, vlined]{algorithm2e}

% Edit these as appropriate
\newcommand\course{Data Science II}
\newcommand\NetID{sliu1@uvm.edu}
\newcommand\Author{Sida Liu}
\pagestyle{fancyplain}
\headheight 35pt
\lhead{\NetID\\\Author}
\chead{\textbf{\Large Assignment \chapternumber }}
\rhead{\course \\ \today}
\lfoot{}
\cfoot{}
\rfoot{\small\thepage}
\headsep 1.5em

\setlength{\parskip}{\baselineskip}%
\setlength{\parindent}{0pt}%

\newenvironment{list_abc}
{ \begin{enumerate}[label=(\alph*)] }
{ \end{enumerate} }

\newenvironment{list_iv}
{ \begin{enumerate}[label=\roman*.] }
{ \end{enumerate} }

\hypersetup{
    colorlinks=true,
    linkcolor=blue,
    filecolor=magenta,      
    urlcolor=cyan,
}
\usepackage{enumerate}
\usepackage{float}

\begin{document}
Collaborators: Julia Zimmerman, Jordan Donovan, Sam Rosenblatt, Milo Trujillo, Phil Nguyen, Nicholas Vartanian,Connor Klopfer, Brett Meyer...
We might have discussed problems related to the assignments via Slack.
\section{Manual Gradient Descent}
\begin{align*}
    \frac{\partial \ell_i}{\partial \theta_j} = -2 x_{ij} (y_i - \sum_{j'=1}^{p} x_{ij'} \theta_{j'}) \\
\end{align*}
\begin{align*}
    \nabla_{\theta} \ell & = \sum_{i=1}^{N} \begin{bmatrix}
        \frac{\partial \ell_i}{\partial \theta_1} \\
        \frac{\partial \ell_i}{\partial \theta_2} \\
        \vdots                                    \\
        \frac{\partial \ell_i}{\partial \theta_p} \\
    \end{bmatrix} \\\\
                         & = -2 x^T (y-x\theta)
\end{align*}

I used the dataset this \href{https://www.kaggle.com/sulianova/cardiovascular-disease-dataset}{Cardiovascular Disease dataset}.

The are 70k records in this dataset.
The task is to predict whether a person has cardiovascular disease or not base on his age, height, etc.
I split the dataset into 80\% training and 20\% testing.

The source code is on \href{https://github.com/liusida/ds2/blob/main/assignment3/code/q1.py}{GitHub q1.py}.

I discovered that larger batch size and larger learning rate will result in faster convergence,
however, if learning rate is too large, the learning might diverge,
and if batch size and learning rate is too large will cause overflow.
So I normalized the input to reduce the possibility of overflow, and choose a batch size of 7k, and a small learning rate.
Now the learning process with 1k epochs takes 1.5 seconds. (If I use a batch size of 100, it will take 9.8 seconds.)

The final test accuracy is 0.643.
I also report the final confusion matrix:
Sensitivity 0.574, Specificity 0.712,
Precision 0.667, Negative Predictive Value 0.625.

Why would I choose to manually do gradient descent?
Because I want to learn how it works, and also, as David mentioned in class, sometimes we have constraints in practice, for example there's possibility that PyTorch is not supported.

\newpage
\section{2D Rosenbrock function}
\begin{figure}[h]
    \includegraphics[width=.99\textwidth]{./rosenbrock.pdf}
    \caption{Visualization of 2D Rosenbrock function. $x \in (-30,30)$, $y \in (-300,800)$}
\end{figure}

It is hard for SGD because the low area is not a dot but a long belt.
If the learning rate is large, it will diverge.
If the learning rate is low, it will be very slow in finding the minima on that flat belt.

Analytically, we need to solve the equations: $\frac{\partial f}{\partial x} = 0$ and $\frac{\partial f}{\partial y} = 0$.

\begin{align*}[left = \empheqlbrace]
    \frac{\partial}{\partial x} (1-x)^2+100(y-x^2)^2 & = 0 \\
    \frac{\partial}{\partial y} (1-x)^2+100(y-x^2)^2 & = 0 \\
\end{align*}

\begin{align*}[left = \empheqlbrace]
    400x^3-400xy+2x-2 & = 0  \\
    y                 & =x^2 \\
\end{align*}

\begin{align*}
    400x^3-400x^3+2x-2 & = 0 \\
    x                  & =1  \\
    y                  & =1  \\
\end{align*}

Using routine gradient descent, start from $x_0=0, y_0=0$, learning rate $\gamma=10^{-3}$, run for 10k steps.
The optimization successfully converged to the minimum.

This is because the initialization is quite close to the minimum,
and because it is smooth in $x \in (0,1)$ and $ y \in (0,1)$.
Figure \ref{fig:routine_gd} shows the trajectory of the optimization.

\begin{figure}[h]
    \includegraphics[width=.8\textwidth]{./routine_gd.png}
    \caption{Visualization of the optimization trajectory. $x \in (0,1)$, $y \in (0,1)$.}
    \label{fig:routine_gd}
\end{figure}

If we are not that lucky and use a bad initialization, such as $x_0 = 3$, $y_0 = -2$,
then the routine gradient descent will be harder.
The learning rate need to be smaller, otherwise it will diverge.
Then momentum is needed to speed the optimization up.
I set momentum to 0.99, learning rate $\gamma=10^{-5}$, and run for 10k steps.
The optimization successfully converged to the minimum.
Figure \ref{fig:gd_m} shows the trajectory.

\begin{figure}[h]
    \includegraphics[width=.8\textwidth]{./gd_m.png}
    \caption{Visualization of the optimization trajectory with momentum. $x \in (0,3)$, $y \in (-2,5)$.}
    \label{fig:gd_m}
\end{figure}

The source code is on \href{https://github.com/liusida/ds2/blob/main/assignment3/code/q2.py}{GitHub q2.py}.

\newpage
\section{Math notations and plate notations}
\subsection{Basic Parametric Density Estimation}
Mathematically, let $l=\text{loc}$, $s=\text{inv scale}$, we have:
\begin{align*}
     & p(l, s, d_i)                                                                                                                                                \\
     & \propto p(l)p(s)\prod_{i=1}^N p(d_i|l,s)                                                                                                                    \\
     & = \text{Normal}(0,1) \text{Inv-Gamma}(3,2) \prod_{i=1}^N \text{Normal}(l, s)                                                                                \\
     & = \frac{1}{\sqrt{2\pi}}e^{-\frac{1}{2}l^2} \frac{2^3}{3!}s^{-3-1} e^{-\frac{2}{s}} \prod_{i=1}^N \frac{1}{s\sqrt{2\pi}} e^{-\frac{1}{2}(\frac{d_i-l}{s})^2} \\
     & = \frac{4}{3} (\frac{1}{\sqrt{2\pi}})^{N+1} s^{-4-N} e^{-\frac{l^2}{2} -\frac{2}{s} - \frac{1}{2}\sum_{i=1}^N (\frac{d_i-l}{s})^2}                          \\
\end{align*}
It doesn't seem to have conjugate prior to me.

Plate notation:
\begin{figure}[h]
    \centering
    \tikz{
        % nodes
        \node[obs] (data) {$d_i$};%
        \node[latent,above=of data] (loc) {loc};
        %    \node[latent,above=of loc,xshift=-1cm] (mu) {$\mu$}; %
        %    \node[latent,above=of loc,xshift=1cm] (sigma) {$\sigma$}; %
        \node[latent,left=of data] (scale) {scale};
        \node[latent,left=of scale] (inv_scale) {inv\_scale};
        %    \node[latent,left=of inv_scale,yshift=1cm] (alpha) {$\alpha$};
        %    \node[latent,left=of inv_scale,yshift=-1cm] (beta) {$\beta$};
        % plate
        \plate [inner sep=.3cm,xshift=.02cm,yshift=.2cm] {plate1} {(data)} {N}; %
        % edges
        %    \edge {sigma,mu} {loc}
        %    \edge {alpha,beta} {inv_scale}
        \edge {inv_scale} {scale}
        \edge {scale,loc} {data}
    }
    \caption{Plate notation}
\end{figure}

\newpage
\subsection{Continuous, nonstationary hidden Markov}

Mathematically, let $l=\text{loc}$, $s=e^{\text{log scale}}$, $o=\text{obs scale}$, we have:
\begin{align*}
     & p(l,s,o,x_{n,t},y_{n,t})                                                                                                                                         \\
     & =p(l)p(s)p(o)\prod_{n=1}^N p(x_{n,0}|l,s) \prod_{t=1}^T p(x_{n,t}|l,x_{n,t-1},s) p(y_{n,t}|x_{n,t}, o)                                                           \\
     & =\text{Normal}(0,1) \text{LogNormal}(0,1) \text{Gamma}(2,2) \prod_{n=1}^N \text{Normal}(l,s) \prod_{t=1}^T \text{Normal}(l+x_{n,t-1},s) \text{Normal}(x_{n,t},o) \\
     & =\frac{1}{\sqrt{2\pi}}e^{-\frac{l^2}{2}} \cdot
    \frac{1}{s\sqrt{2\pi}}e^{-\frac{\log(s)^2}{2}} \cdot
    \frac{1}{2!2^2}o^{2-1}e^{-\frac{o}{2}}  \cdot
    \prod_{n=1}^N
    \frac{1}{s\sqrt{2\pi}}e^{-\frac{(x_{n,0}-l)^2}{2s^2}}
    \prod_{t=1}^T
    \frac{1}{s\sqrt{2\pi}}e^{-\frac{(x_{n,t}-l-x_{n,t-1})^2}{2s^2}}
    \frac{1}{o\sqrt{2\pi}}e^{-\frac{(y_{n,t}-x_{n,t})^2}{2o^2}}                                                                                                         \\
\end{align*}

Plate notation (Reference: \href{https://davidrushingdewhurst.com/blog/2020-07-28keep-using-plate-notation.html}{this blog post}):
\begin{figure}[h]
    \centering
    \tikz{
        % \draw[lightgray,ultra thin] (-6,-6) grid (6,6);
        % nodes
        \node[obs] (y_t) {$y_{n,t}$};%
        \node[latent,left=of y_t] (x_t) {$x_{n,t}$};%
        \node[latent,above=of x_t] (x_0) {$x_{n,0}$};%
        \node[latent,above=of x_0,xshift=1.5cm] (loc) {loc};%
        \node[latent,right=of loc] (scale) {scale};%
        \node[latent,above=of scale] (log_scale) {log scale};%
        \node[latent,right=of y_t] (obs_scale) {obs scale};%
        % plate
        \plate [inner sep=.3cm,xshift=.02cm,yshift=.2cm] {episode} {(x_t)(y_t)} {T}; %
        \plate [inner sep=.3cm,xshift=.02cm,yshift=.2cm] {plate1} {(x_0)(episode)} {N}; %
        % edges
        \edge {obs_scale,x_t} {y_t}
        \edge {loc,scale} {x_t}
        \edge {loc,scale} {x_0}
        \edge {log_scale} {scale}
        % hmm...
        \draw[blue, thick, ->,shorten >=1pt] (x_0) to [out=-90,in=90] node[left,yshift=.2cm,xshift=-.2cm] {$t=1$} (x_t);
        \Cycle[blue, ->]{x_t}{180}{7mm}[{node[anchor=0,pos=0.5]{$t=2,...,T (t-1\rightarrow t)$}}]{7mm}
    }
    \caption{Plate notation}
\end{figure}

\newpage
\subsection{Switching model}
Mathematically, let $s'=\text{log scale}$, $s=e^{s'}$, $w=\text{switch}$, $m_1=\text{model1}$, $m_2=\text{model2}$, we have:
\begin{align*}
     & p(s,z_0,z_t,p_t,m_{1,t},m_{2,t},w_{t,n},y_{t,n},x_{t,n})                                                                                             \\\\
     & =p(s)p(z_0|s) \times                                                                                                                                 \\
     & \prod_{t=1}^T p(z_t|z_{t-1}) p(p_t|z_t) 
     \prod_{n=1}^N p(w_{t,n}|p_t) 
     p(m_{1,t}|t) p(m_{2,t}|t) p(y_{t,n}|w_{t,n},m_{1,t},m_{2,t}) 
     p(x_{t,n},y_{t,n}) \\\\
     & =\text{LogNormal}(0,1)
    \text{Normal}(0,s) \times                                                                                                                               \\
     & \prod_{t=1}^T \text{LogitNormal}(z_{t-1},1)
    \prod_{n=1}^N \text{Bernoulli}(p_t)
    p(m_{1,t}|t) p(m_{2,t}|t)
    p(y_{t,n}|w_{t,n},m_{1,t},m_{2,t})
    \text{Poisson}(y_{t,n})                                                                                                                                 \\\\
     & =\frac{1}{s\sqrt{2\pi}}e^{-\frac{\log(x)^2}{2}} \cdot
    \frac{1}{s\sqrt{2\pi}}e^{-\frac{z_0^2}{2s^2}} \cdot \\
    &\prod_{t=1}^T \frac{1}{\sqrt{2\pi}} e^{-\frac{\text{logit}(z_t-z_{t-1})^2}{2}} \cdot
    \prod_{n=1}^N p_t^{w_{t,n}} (1-p_t)^{1-w_{t,n}}
    p(m_{1,t}|t)^{w_{t,n}} p(m_{2,t}|t)^{1-w_{t,n}}
    \frac{y_{t,n}^{x_{t,n}} e^{-y_{t,n}}}{x_{t,n}!}
\end{align*}


Plate notation (next page):
\begin{figure}[!t]
    \centering
    \tikz{
        % \draw[lightgray,ultra thin] (-6,-6) grid (6,6);
        % nodes
        \node[obs] (x_t) {$x_{n,t}$};%
        \node[latent, left=of x_t] (y_t) {$y_{n,t}$};
        \node[latent, below=of y_t, xshift=-2cm] (model1) {model1$_t$};
        \node[latent, below=of y_t] (model2) {model2$_t$};
        \node[latent, left=of y_t, yshift=2cm] (switch) {switch$_{t,n}$};
        \node[latent, left=of switch] (p) {$p_t$};
        \node[latent, left=of p] (z) {$z_t$};
        \node[latent, above=of z] (z0) {$z_0$};
        \node[latent, above=of z0] (scale) {scale};
        \node[latent, left=of scale] (log_scale) {log scale};
        % plate
        \plate [inner sep=.3cm] {plate_switch} {(x_t)(y_t)(switch)} {N};
        \plate [inner sep=.3cm] {episode} {(plate_switch)(model1)(model2)(p)(z)} {T};
        % edges
        \edge {scale} {z0};
        \edge {log_scale} {scale};
        \edge {z} {p};
        \edge {p} {switch};
        \edge {switch, model1, model2} {y_t};
        \edge {y_t} {x_t};
        \draw[blue, thick, ->,shorten >=1pt] (z0) to [out=-90,in=90] node[left,yshift=.2cm,xshift=-.2cm] {$t=1$} (z);
        \Cycle[blue, ->]{z}{180}{7mm}[{node[anchor=0,pos=0.5]{$t=2,...,T (t-1\rightarrow t)$}}]{7mm}
    }
    \caption{Plate notation}
\end{figure}

\newpage
\section{PPL via Pyro}

The source code is on GitHub:
\begin{enumerate}
    \item \href{https://github.com/liusida/ds2/blob/main/assignment3/code/q4.1.py}{q4.1.py}.
    \item \href{https://github.com/liusida/ds2/blob/main/assignment3/code/q4.2.py}{q4.2.py}.
    \item \href{https://github.com/liusida/ds2/blob/main/assignment3/code/q4.3.py}{q4.3.py}.
\end{enumerate}

\newpage
\section{Project Thoughts}
\subsection*{Bayesian Meta Unsupervised Learning}
Suppose we have an intelligent agent.
The agent has a camera and several motors.
If we want the agent to learn its surroundings by itself in an unsupervised fashion,
we might want the agent can learn to processing the video images by itself.
How can we do that?
We can assume that the world is not always changing abruptly.
With this assumption, the agent can update its belief at each time step using Bayesian Theorem.

In this course project, I want to demonstrate the possibility of this in a simplified version.
I am going to use MNIST to do this.
First, construct a small neural network;
Pass one image of ``0'' to the network, we will get a probability $p_0'$;
Pass the second image of ``0'' to the network, we will get the second probability $p_0''$;
$p_0'$ is the prior, and $p_0''$ is the posterior.
So, we should update $p_0'$ into $\frac{p_0' p_0''}{z}$, where $z$ is the normalization factor.
So, the loss function can be:
\begin{align*}
    \mathcal{L} = ||p_0' - \frac{p_0' p_0''}{z}||_2^2
\end{align*}
Then this learning signal can be used to adjust weights of the networks.

After we trained the network with ``0''s for a while, we can train ``1''s, ``2''s, and so on.
We ignore the learning signal during switching between different numbers (because in the agent scenario, we assume there's no such abrupt change, the world is changing gradually.)

After all training is done, we can pass the test images through the network and obtain the probabilities, 
and we can measure if the outcomes are different when the numbers are different.

I haven't look into the literatures for this idea.
I am not sure it is novel or not.
\end{document}
